\section{Jarosław Klima}
\label{sec:jk}

\subsection{Wyrażenie matematyczne}
Wzór: $$V = \frac{s}{t} [\frac{metr}{sekunda}]$$

\subsection{Fotografia}
Fotografia Fortnite:
\begin{figure}[htbp]
    \centering
    \includegraphics[width=0.6\linewidth]{pictures/JaroslawKlimaPic.jpg}
    \caption{Przykładowa fotografia.}
    \label{fig:fortnite}
\end{figure}

\subsection{Tablica}
Tabliczka mnożenia:
\begin{table}[]
\centering
\begin{tabular}{|l|lllllllllllllll}
\hline
\multicolumn{1}{|c|}{X} & \multicolumn{1}{l|}{1} & \multicolumn{1}{l|}{2} & \multicolumn{1}{l|}{3} & \multicolumn{1}{l|}{4} & \multicolumn{1}{l|}{5} & \multicolumn{1}{l|}{6} & \multicolumn{1}{l|}{7} & \multicolumn{1}{l|}{8} & \multicolumn{1}{l|}{9} & \multicolumn{1}{l|}{10} & \multicolumn{1}{l|}{11} & \multicolumn{1}{l|}{12} & \multicolumn{1}{l|}{13} & \multicolumn{1}{l|}{14} & \multicolumn{1}{l|}{15} \\ \hline
1                       & 1                      & 2                      & 3                      & 4                      & 5                      & 6                      & 7                      & 8                      & 9                      & 10                      & 11                      & 12                      & 13                      & 14                      & 15                      \\ \cline{1-1}
2                       & 2                      & 4                      & 6                      & 8                      & 10                     & 12                     & 14                     & 16                     & 18                     & 20                      & 22                      & 24                      & 26                      & 28                      & 30                      \\ \cline{1-1}
3                       & 3                      & 6                      & 9                      & 12                     & 15                     & 18                     & 21                     & 24                     & 27                     & 30                      & 33                      & 36                      & 39                      & 42                      & 45                      \\ \cline{1-1}
4                       & 4                      & 8                      & 12                     & 16                     & 20                     & 24                     & 28                     & 32                     & 36                     & 40                      & 44                      & 48                      & 52                      & 56                      & 60                      \\ \cline{1-1}
5                       & 5                      & 10                     & 15                     & 20                     & 25                     & 30                     & 35                     & 40                     & 45                     & 50                      & 55                      & 60                      & 65                      & 70                      & 75                      \\ \cline{1-1}
6                       & 6                      & 12                     & 18                     & 24                     & 30                     & 36                     & 42                     & 48                     & 54                     & 60                      & 66                      & 72                      & 78                      & 84                      & 90                      \\ \cline{1-1}
7                       & 7                      & 14                     & 21                     & 28                     & 35                     & 42                     & 49                     & 56                     & 63                     & 70                      & 77                      & 84                      & 91                      & 98                      & 105                     \\ \cline{1-1}
8                       & 8                      & 16                     & 24                     & 32                     & 40                     & 48                     & 56                     & 64                     & 72                     & 80                      & 88                      & 96                      & 104                     & 112                     & 120                     \\ \cline{1-1}
9                       & 9                      & 18                     & 27                     & 36                     & 45                     & 54                     & 63                     & 72                     & 81                     & 90                      & 99                      & 108                     & 117                     & 126                     & 135                     \\ \cline{1-1}
10                      & 10                     & 20                     & 30                     & 40                     & 50                     & 60                     & 70                     & 80                     & 90                     & 100                     & 110                     & 120                     & 130                     & 140                     & 150                     \\ \cline{1-1}
11                      & 11                     & 22                     & 33                     & 44                     & 55                     & 66                     & 77                     & 88                     & 99                     & 110                     & 121                     & 132                     & 143                     & 154                     & 165                     \\ \cline{1-1}
12                      & 12                     & 24                     & 36                     & 48                     & 60                     & 72                     & 84                     & 96                     & 108                    & 120                     & 132                     & 144                     & 156                     & 168                     & 180                     \\ \cline{1-1}
13                      & 13                     & 26                     & 39                     & 52                     & 65                     & 78                     & 91                     & 104                    & 117                    & 130                     & 143                     & 156                     & 169                     & 182                     & 195                     \\ \cline{1-1}
14                      & 14                     & 28                     & 42                     & 56                     & 70                     & 84                     & 98                     & 112                    & 126                    & 140                     & 154                     & 168                     & 182                     & 196                     & 210                     \\ \cline{1-1}
15                      & 15                     & 30                     & 45                     & 60                     & 75                     & 90                     & 105                    & 120                    & 135                    & 150                     & 165                     & 180                     & 195                     & 210                     & 225                     \\ \cline{1-1}
16                      & 16                     & 32                     & 48                     & 64                     & 80                     & 96                     & 112                    & 128                    & 144                    & 160                     & 176                     & 192                     & 208                     & 224                     & 240                     \\ \cline{1-1}
17                      & 17                     & 34                     & 51                     & 68                     & 85                     & 102                    & 119                    & 136                    & 153                    & 170                     & 187                     & 204                     & 221                     & 238                     & 255                     \\ \cline{1-1}
18                      & 18                     & 36                     & 54                     & 72                     & 90                     & 108                    & 126                    & 144                    & 162                    & 180                     & 198                     & 216                     & 234                     & 252                     & 270                     \\ \cline{1-1}
19                      & 19                     & 38                     & 57                     & 76                     & 95                     & 114                    & 133                    & 152                    & 171                    & 190                     & 209                     & 228                     & 247                     & 266                     & 285                     \\ \cline{1-1}
20                      & 20                     & 40                     & 60                     & 80                     & 100                    & 120                    & 140                    & 160                    & 180                    & 200                     & 220                     & 240                     & 260                     & 280                     & 300                     \\ \cline{1-1}
\end{tabular}
\label{tab:mnozenie_jk}
\end{table}

\subsection{Lista numerowana}
Przykład listy numerowanej:
\begin{enumerate}
  \item Pierwszy punkt.
  \item Drugi punkt.
  \item Trzeci punkt.
\end{enumerate}

\subsection{Lista nienumerowana}
Przykład listy nienumerowanej:
\begin{itemize}
  \item Kolejny punkt.
  \item ostatni punkt.  
\end{itemize}

\subsection{Tekst}
\par \textbf{\underline{Tekst w tym akapicie jest pogrubiony oraz podkreślony. Tekst w tym akapicie jest pogrubiony oraz podkreślony. Tekst w tym akapicie jest pogrubiony oraz podkreślony. Tekst w tym akapicie jest pogrubiony oraz podkreślony.}} \par \textit{Tekst w tym akapicie jest napisany w kursywie. Tekst w tym akapicie jest napisany w kursywie. Tekst w tym akapicie jest napisany w kursywie. Tekst w tym akapicie jest napisany w kursywie. Tekst w tym akapicie jest napisany w kursywie.} 
