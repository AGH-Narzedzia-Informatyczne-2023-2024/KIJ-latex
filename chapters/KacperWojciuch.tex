\section{Kacper Wojciuch}
\label{sec:kw}

\subsection{Wyrażenie matematyczne}
Losowa całka z zadań z analizy (zestaw 6, ćw. 5, przykład e):
\[\int_{5}^{+\infty} \frac{x \,dx}{\sqrt{x^5 - 3}}\]

\subsection{Lista nienumerowana}
Przedmioty na pierwszym semestrze:
\begin{itemize}
    \item Algebra liniowa
    \item Analiza matematyczna
    \item Wychowanie fizyczne
    \item Narzędzia informatyczne
    \item Architektury komputerów
    \item Podstawy programowania
    \item Logika dla informatyków
    \item Wstęp do informatyki
    \item Wstęp do systemów UNIX-owych
\end{itemize}

\subsection{Lista numerowana}
Kolejność komend przy tworzeniu commita:
\begin{enumerate}
    \item \verb|git add *|
    \item \verb|git commit -m "Small changes"|
    \item \verb|git push|
\end{enumerate}

\subsection{Tekst z akapitami}
Trochę \textit{lorem ipsum}:

\textbf{Lorem ipsum} dolor sit amet, consectetur adipiscing elit. Vestibulum ac mollis nisi, vitae viverra orci. Donec dictum turpis nec arcu accumsan, sollicitudin eleifend sem efficitur. Morbi rhoncus nisl in odio fringilla malesuada. Sed a tincidunt ligula, sit amet bibendum diam. Interdum et malesuada fames ac ante ipsum primis in faucibus. \underline{Suspendisse} gravida porttitor lorem, porta venenatis lectus sagittis eget. Nulla eget massa enim. Vestibulum a leo non purus suscipit feugiat. Cras ut dictum nulla. Nam vestibulum velit lorem, eu tempus arcu iaculis in. In tempor neque sed est placerat, non porta turpis commodo.

\textit{Integer dolor ligula, venenatis ac varius sit amet, pulvinar eu lorem.} Suspendisse potenti. Nulla commodo, nisi id sagittis mattis, sem dui vehicula nunc, sit amet scelerisque orci leo vitae lectus. Etiam vitae nisl at leo malesuada semper. Suspendisse feugiat vel nisl in porttitor. Maecenas maximus dolor vel risus imperdiet, vitae tempor quam elementum. Praesent imperdiet auctor augue quis vehicula. Fusce sed massa posuere, laoreet justo a, condimentum tortor.

\subsection{Obrazek}
Odwołanie do obrazka kota: zobacz Rysunek~\ref{fig:miauczka}.
\begin{figure}[htbp]
    \centering
    \includegraphics[width=0.7\textwidth]{pictures/KacperWojciuch.png}
    \caption{Mój kot, Miauczka}
    \label{fig:miauczka}
\end{figure}

\subsection{Tabela}
Odwołanie do poniższej planszy: zobacz Tabelę~\ref{tab:szachy}.
% Użyłem sugerowanego generatora

\begin{table}[htbp]
\centering
\begin{tabular}{|cccccccc|}
\hline
\multicolumn{8}{|c|}{\textbf{\begin{tabular}[c]{@{}c@{}}Plansza do szachów\\ (perspektywa białych)\end{tabular}}}                                                                 \\ \hline
\multicolumn{1}{|c|}{r} & \multicolumn{1}{c|}{n} & \multicolumn{1}{c|}{b} & \multicolumn{1}{c|}{q} & \multicolumn{1}{c|}{k} & \multicolumn{1}{c|}{b} & \multicolumn{1}{c|}{n} & r \\ \hline
\multicolumn{1}{|c|}{p} & \multicolumn{1}{c|}{p} & \multicolumn{1}{c|}{p} & \multicolumn{1}{c|}{p} & \multicolumn{1}{c|}{p} & \multicolumn{1}{c|}{p} & \multicolumn{1}{c|}{p} & p \\ \hline
\multicolumn{1}{|c|}{}  & \multicolumn{1}{c|}{}  & \multicolumn{1}{c|}{}  & \multicolumn{1}{c|}{}  & \multicolumn{1}{c|}{}  & \multicolumn{1}{c|}{}  & \multicolumn{1}{c|}{}  &   \\ \hline
\multicolumn{1}{|c|}{}  & \multicolumn{1}{c|}{}  & \multicolumn{1}{c|}{}  & \multicolumn{1}{c|}{}  & \multicolumn{1}{c|}{}  & \multicolumn{1}{c|}{}  & \multicolumn{1}{c|}{}  &   \\ \hline
\multicolumn{1}{|c|}{}  & \multicolumn{1}{c|}{}  & \multicolumn{1}{c|}{}  & \multicolumn{1}{c|}{}  & \multicolumn{1}{c|}{}  & \multicolumn{1}{c|}{}  & \multicolumn{1}{c|}{}  &   \\ \hline
\multicolumn{1}{|c|}{}  & \multicolumn{1}{c|}{}  & \multicolumn{1}{c|}{}  & \multicolumn{1}{c|}{}  & \multicolumn{1}{c|}{}  & \multicolumn{1}{c|}{}  & \multicolumn{1}{c|}{}  &   \\ \hline
\multicolumn{1}{|c|}{P} & \multicolumn{1}{c|}{P} & \multicolumn{1}{c|}{P} & \multicolumn{1}{c|}{P} & \multicolumn{1}{c|}{P} & \multicolumn{1}{c|}{P} & \multicolumn{1}{c|}{P} & P \\ \hline
\multicolumn{1}{|c|}{R} & \multicolumn{1}{c|}{N} & \multicolumn{1}{c|}{B} & \multicolumn{1}{c|}{Q} & \multicolumn{1}{c|}{K} & \multicolumn{1}{c|}{B} & \multicolumn{1}{c|}{N} & R \\ \hline
\end{tabular}
\caption{Pozycja startowa w szachach}
\label{tab:szachy}
\end{table}

\subsection{Integracja z GitHubem}
Ta linijka została dodana lokalnie z mojego komputera poprzez Gita.

A z kolei ta zmiana została wykonana już po zaimportowaniu commita na Overleafa.
